
\documentclass{article}
\title{Documentation: NER Component}
\author{Tim Pietz (6808046)\and Inga Kempfert (6824114)}
\date{\today}

\usepackage{csquotes}
\usepackage[margin=1.5in]{geometry}
\usepackage{footnote}
\usepackage{hyperref}
\usepackage[utf8]{inputenc}
\usepackage[T1]{fontenc}

\begin{document}
\maketitle

\section{Task Description}
The task was to implement a Named Entity Recognition component in UIMA using a conditional random field (CRF).
This component is supposed to label named entities such as persons, locations and organisations in input text.

\section{Implementation}
We used the UIMA framework to implement a NER component based on the POS tagger given as an example.

\subsection{The Pipeline}
The pipeline consists of one reader and four analysis engines.
The reader reads the raw CONLL-formated input data into the CONLL jCas view.

The first analysis engine \texttt{NERReader} parses the CONLL view into tokens, POS tags and NEIOBAnnotations.
The latter represents the tags for a named entity.
For example, a named entity such as \enquote{Allianz Versicherungen} would be annotated as two separate consecutive annotations of the type \enquote{I-ORG}.
The NEIOBAnnotations contain the gold value as well as value, which is predicted by the CRF.
We included the POS tags, that are annotated in the input data, as separate POS annotations in the \texttt{NERReader}.

The second analysis engine -- the Snowball Stemmer -- stems the input data.

The third analysis engine is the \texttt{KnownNEAnnotator}, which we implemented ourselves.
It extracts named entities from external resources and annotates those to the input data using our own annotation type called \texttt{KnownNE}.

The fourth analysis engine is the \texttt{NERAnnotator}, which does the main part the named entity recognition.
It extracts the relevant features from the input data using the ClearTK feature format and feature extractors.
It passes the extracted features to ClearTK which either trains the model or classifies data using the model.

During testing phase, another analysis engine is included in the pipeline: \texttt{ScoreNER}.
It writes the gold and predicted value in order to be able to evaluate the performance of the NER component.

\subsection{External Resources}

% knownNEAnnotator
There are a few relevant classes in the NER component.
We implemented out own annotator \texttt{KnownNEAnnotator}, which is responsible for annotating known named entities in the input data.
A \emph{known named entity} is a named entity that can be found in external resources.
The external resources occur in different file formats, each of which is handled by a specifically designed file reader. Currently, we support readers for three different file formats:
\begin{itemize}
	\item The \texttt{KNEListSupplier} handles the format found in \texttt{eng.list} and \texttt{name.list} which is shown below.
	Both of the files were given.
	\begin{verbatim}
	PER Inspector Maigret
	LOC Africa
	MISC Vietnam War
	ORG Eletropaulo
	\end{verbatim}
	The data can be split into two columns.
	The first column contains the NER tag, while the second column contains the named entity name.
	\item The \texttt{jrcKNESupplier} handles the format found in the list \texttt{entities}\footnote{\url{https://ec.europa.eu/jrc/en/language-technologies/jrc-names}}.
	An example for the data format is shown below:
	\begin{verbatim}
	1467646	P	u	Habiba+Ghribi
	1467688	P	u	Melanie+Schultz+van+Infrastructuur
	65795	O	u	Streitkräfte+Serbiens
	858251	O	u	Partei+fur+Gerechtigkeit+und+Aufschwung
	\end{verbatim}
	For our purposes, only the second and the fourth column are relevant: the second column contains the type of named entity(P=Person, O=Organisation) and the fourth column contains the named entity name.
	\item The \texttt{geonameKNESupplier} handles the format found in the list \texttt{cities500.txt}\footnote{\url{http://download.geonames.org/export/dump/}}.
	An example for the data format is shown below:
	\begin{verbatim}
	4061113	Excel Excel	Excel,Excel Station	31.42794 -87.34137 P
	PPL US	AL 099 672 124 128 America/Chicago 2017-03-09
	\end{verbatim}
	For our purposes, only the second column is relevant.
	It is always assumed that entries from this list are locations.
\end{itemize}
The combined output of the KNESuppliers is stored in a trie structure for efficient search in larger documents.
Each occurrence of a named entity found in the document text is annotated by a KNEAnnotation of the respective type (PERS, LOC, MISC or ORG).

\subsection{Features}
% NERAnnotator
The \texttt{NERAnnotator} extracts features tokenwise from the input data and passes them to the model instances. We use the CleartkExtracter to extract a set of features for a window of five tokens: the two proceeding tokens, the current one and the two following tokens, in order to include some context. For each token we extract the following features:
\begin{itemize}
\item the text covered by the token
\item whether the token is all uppercase, all lowercase, initial uppercase or mixed case
\item character 3-gram for the token suffix
\item character 3-gram for the token prefix
\item the character category pattern function
\item the token stem
\item the POS tag
\item the KnownNEAnnotation
\end{itemize}
Since the KNEAnnotations are added on a per-entity basis as opposed to a per-token basis, we build up an index using \texttt{JCasUtil.indexCovering} to quickly find the KNEAnnotations that cover a given token.
This index is passed to our \texttt{RealtionIndexExtractor} which, given a single Token, applies both the token and each covering KnownNEAnnotation to our \texttt{KnownNeExtractor} which implements \texttt{FeatureExtractor2<Token, KnownNEAnnotation>}. The KnownNeExtractor adds features of the form like \enquote{KNE-*}, where * is PERS, ORG, LOC or MISC.


\section{Evaluation}


% ScoreNER
% on the dev set

% different feature combis

% which features work well which don't


\end{document}